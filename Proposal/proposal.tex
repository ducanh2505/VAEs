%Đây là template dùng cho đề cương đề tài tốt nghiệp
%Khoa Công nghệ Thông tin
%Trường Đại học Khoa học Tự nhiên, ĐHQG-HCM

%Liên hệ về mẫu LaTEX này: Thầy Bùi Huy Thông (bhthong@fit.hcmus.edu.vn)

\documentclass{article}[14pt]
\usepackage[utf8]{vietnam}
\usepackage{enumerate}
\usepackage{enumitem}
\usepackage{multicol}
\usepackage{listings}
\usepackage[left=2cm,right=2cm,top=2.5cm,bottom=2.5cm]{geometry}
\usepackage{verbatim}
\usepackage{graphicx}
\usepackage{url}
\usepackage{fancyhdr}
\usepackage{fancybox,framed}
\linespread{1.2}
\usepackage{lastpage}
\usepackage{floatrow}
\usepackage{floatrow}
\pagenumbering{arabic}
%\pagestyle{fancy}
\newfloatcommand{capbtabbox}{table}[][\FBwidth]

\usepackage{blindtext}
\usepackage{titlesec}
\usepackage[nottoc]{tocbibind}
\usepackage{array}
\titleformat*{\section}{\LARGE\bfseries}
\titleformat*{\subsection}{\Large\bfseries}
\titleformat*{\subsubsection}{\large\bfseries}
%\addbibresource{ref.bib}


\begin{document}
% \begin{obeylines}
    \begin{figure}[h]
        \begin{floatrow}
        \ffigbox{\includegraphics[scale = 0.1]{logo.png}}  
        {%
    
        }
        \capbtabbox{
            \begin{tabular}{l}
            \multicolumn{1}{c}{\textbf{\begin{tabular}[c]{@{}c@{}}TRƯỜNG ĐẠI 
HỌC KHOA HỌC TỰ NHIÊN\\KHOA CÔNG NGHỆ THÔNG TIN\end{tabular}}} 
\\ \\ \\
            \end{tabular}
        }
        {%
    
        }
        \end{floatrow}
    \end{figure}
    
    \begin{center}
        
        %Xác định loại đề tài tốt nghiệp tương ứng: Khóa luận, Thực tập, Đồ án
        \textbf{\Large ĐỀ CƯƠNG KHOÁ LUẬN TỐT NGHIỆP} \\ 
    \end{center}
    
    %\vspace{.5cm}
    
    \begin{center}
    %Tên đề tài phải VIẾT HOA
        
        \textbf{\huge  XÂY DỰNG HỆ THỐNG GỢI Ý SẢN PHẨM BẰNG PHƯƠNG PHÁP HỌC 
SÂU} \\
        
    %Tên đề tài bằng tiếng Anh (nếu có)
    \vspace{.5cm}
        \textit{\textbf{\Large (Building Recommendation System using Deep 
Learning)}}
    \end{center}
    
    \vspace{.5cm}
    
    \Large
    \section{THÔNG TIN CHUNG}
    \begin{itemize}[label = {}]
        
        \item \textbf{Người hướng dẫn:} 
        %Thể hiện dạng: <Chức danh> <Họ và tên> (<Đơn vị công tác>)
        \begin{itemize}
            % \item TS. Trần Văn A (Khoa Công nghệ Thông tin)
            % \item Bà Lý Thị B (Công ty XYZ)
            \item ThS. Trần Trung Kiên (Khoa Công nghệ Thông tin)
        \end{itemize}{}
    
        
        \item \textbf{Nhóm sinh viên thực hiện:}
        
        %Thể hiện dạng: <Họ và tên sinh viên> (MSSV: )
        \begin{enumerate}
        
            \item Đào Đức Anh (MSSV: 1712270)
            \item Nguyễn Thành Nhân (MSSV: 1712631)
            
        \end{enumerate}

       %Chọn loại thích hợp
        \item \textbf{Loại đề tài:} Nghiên cứu
        
        \item \textbf{Thời gian thực hiện:} Từ \textit{1/2021} đến 
\textit{6/2021}
        
        
    \end{itemize}
    
    \section{NỘI DUNG THỰC HIỆN}
    {

    %Mỗi mục dưới đây phải viết ít nhất là 5 câu mô tả/giới thiệu.
    
    \subsection{Giới thiệu về đề tài}
    
    % Phần này giới thiệu về đề tài và ngữ cảnh thực hiện đề tài.
    
      % Phần này giới thiệu về đề tài và ngữ cảnh thực hiện đề tài.
    
      Hệ thống gợi ý sản phẩm (recommendation system) là một lớp con của hệ thống
      lọc thông tin thường được tích hợp trên các trang web nhằm hỗ trợ 
      người dùng (user) tìm kiếm được đúng sản phẩm (item) cần thiết  
      và tìm cách dự đoán \textit{xếp hạng} hoặc \textit{sở thích} 
      của họ cho một sản phẩm.
      
      Các nội dung nói trên được gọi là các gợi ý, được tính toán thông qua dữ liệu
      của người dùng về các sản phẩm đó như lịch sử mua hàng, phản hồi, đánh giá của
      người dùng, ...
  
      Có hai cách tiếp cận chính để xây dựng hệ thống gợi ý là 
      \textit{Collaborative filtering} và \textit{Content-based filtering}.
      Trong đó, \textit{Collaborative filtering} là hướng tiếp cận dựa trên các 
      tương tác của người dùng với sản phẩm trong quá khứ để đưa ra các gợi ý. 
      \textit{Content-base filtering} là hướng tiếp cận dựa trên các thông tin 
      của người dùng như: độ tuổi, giới tính, công việc, ... và các thông tin 
      của sản phẩm như: phân loại, khối lượng, giá tiền, ... để đưa ra các gợi ý.
  
      Collaborative filtering là hướng tiếp cận được sử dụng rộng rãi trong các 
      hệ thống gợi ý. 
      Mô hình gợi ý sản phẩn dựa trên Collaborative filtering được phát biểu như 
      sau:
      \begin{itemize}
          \item Cho input là lịch sử tương tác của người dùng với các item:
              số lượt nhấp chuột, ấn thích, đánh giá, ... 
          \item Yêu cầu: đưa ra tập các item (không có trong lịch sử) 
          được dự đoán là phù hợp với người dùng (tự động làm bằng máy).
      \end{itemize}
  
      Trong thời gian gần đây, các nghiên cứu chỉ ra rằng việc áp dụng mạng 
      nơ-ron nhân tạo (neural network) trong việc cài đặt cho Collaborative filtering 
      mang lại những tín hiệu tích cực. 
      Và đây cũng là hướng tiếp cận mà chúng em chọn để tìm hiểu.
  
      \subsection{Mục tiêu đề tài}
      
      % Phần này mô tả mục tiêu thực hiện đề tài.
  \begin{itemize}
  \item     Hiểu rõ được tình hình nghiên cứu của bài toán xây dựng hệ thống gợi ý
   hiện nay (biết được các hướng tiếp cận phổ biến đồng thời cũng như là ý tưởng 
   và ưu nhược điểm của các hướng tiếp cận đó; ngoài ra còn nắm được các 
   thách thức và thuận lợi trong việc giải quyết bài toán gợi ý sản phẩm). Từ cơ 
   sở đó có thể chọn ra một hướng tiếp cận phù hợp để tìm hiểu sâu và thực hiện 
   cài đặt mô hình theo hướng đã chọn.
  \item      Nắm rõ các kiến thức nền tảng bên dưới (toán học, xác suất thống kê, 
  học máy, ...) của mô hình đã chọn.
  \item     Cài đặt lại mô hình  để đạt được kết quả như trong bài báo tương ứng; 
  có thể tiến hành thêm các thí nghiệm ngoài báo cáo để thấy rõ hơn về ưu nhược 
  của mô hình.
  \item     Trên cơ sở kiến thức nắm được từ mô hình có thể thực hiện các tối ưu 
  (về kết quả, tốc độ huấn luyện, ...).
  \item     Rèn luyện các kĩ năng: suy nghĩ rõ ràng, lên kế hoạch làm việc, 
  làm việc theo nhóm, khả năng trình bày báo cáo, thuyết trình, ...
  \end{itemize}

    
    \subsection{Phạm vi của đề tài}
    
    % Phần này giới hạn phạm vi thực hiện của đề tài.
    Đề tài làm với dữ liệu có các phản hồi ẩn của người dùng cho một sản 
phẩm; cụ thể, chúng em dự kiến sẽ làm với bộ dữ liệu thường được sử  
dụng trong lĩnh vực xây dựng  hệ thống gợi ý là MovieLens.
    Về cơ bản, đề tài chỉ tìm hiểu và cài đặt lại mô hình của một bài báo uy 
tín, ngoài ra có thể thêm các thí nghiệm khác cũng như huấn luyện trên 
bộ dữ liệu khác ngoài bài báo để thấy rõ hơn về ưu, nhược điểm của mô 
hình.
    Lý do chúng em giới hạn đề tài như vậy là vì: (i) chúng em thấy riêng 
việc hiểu rõ mô hình (và các kiến thức nền tảng bên dưới) và có thể tự 
cài đặt lại đã tốn khá nhiều thời gian, và (ii) chúng em xác định là chỉ 
trên cơ sở hiểu rõ mô hình (và các kiến thức nền tảng bên dưới) thì mớ i 
có thể có được các cải tiến thật sự trong tương lai, cũng như là có thể 
vận dụng mô hình được cho các bài toán khác.
    Tất nhiên, trong khóa luận, nếu có đủ thời gian thì chúng em sẽ thử đề 
xuất và cài đặt các cải tiến; tuy nhiên, chúng em xác định đây không 
phải là mục đích chính.
    
    \subsection{Cách tiếp cận dự kiến}
    
    %Có thể bổ sung hình ảnh vào để làm rõ phương pháp hoặc cách tiếp cận dự 
kiến.
    Sau đây sẽ là trình bày một số mô hình theo hướng tiếp cận sử dụng các 
mô hình học sâu để giải quyết bài toán xây dựng hệ thống gợi ý sản phẩm 
mà chúng em đã tìm hiểu được cho đến thời điểm hiện tại, cũng như là mô 
hình mà chúng em dự kiến sẽ chọn để tập trung tìm hiểu sâu. 
    
    
\begin{itemize}
\item  	“Neural collaborative filtering” của Xiangnan He cùng các cộng sự 
được công bố tại hội nghị WWW’] \cite{neumf} (số lần trích dẫn là 1967 tính 
cho đến thời điểm hiện tại là ngày 28/2/2020) là một trong những bài báo nổi 
bật được coi như là bước đặt nền móng cho việc áp dụng mạng nơ ron kết hợp 
với phương pháp truyền thống collaborative để xẩy dựng một hệ thống gợi ý 
sản phẩm.
Điểm nổi bật của mô hình này là tận dụng sức mạnh của các mạng học sâu để 
đánh bật các thuật đoán tại thời điểm đó như Matrix Factorization, …  đã 
trích xuất được các đặc trưng phi tuyến từ đó mô hình có khả năng mô hình 
hóa dữ liệu tương tác trong quá khứ của người dùng từ đó trích xuất các đặc 
trưng ẩn giữa các user và các item để cải thiện độ chính xác cho hệ thống 
gợi ý.
Mô hình được đề xuất từ nhóm tác giả gồm 2 phần: 
\begin{itemize}
    \item  Generalized Matrix factorization viết tắt GMF là việc kết hợp 
nhân linear kernel cho với phương pháp Matrix factorization để học những 
các tương tác ẩn giữa tập người dùng và user (latent interaction).
    Trả về các embeding vector thể hiện cho người dùng và các item
    \item 	Mạng nơ ron nhiều tầng MLP (Multi layer Perceptron) để đưa ra 	
kết quả từ embeding vector có được từ GMF
\end{itemize}
Mô hình “NeuMF” này được huấn luyện với bộ dữ liệu implicit feedback bao gồm 
các vector one hot coding thể hiện tương tác giữa user và các item.
Mô hình này tuy đơn giản sử dụng MLP nhưng đã học được các đặc trưng phi 
tuyến thông qua mạng nơ ron chứ không cần trích xuất bằng tay như những công 
trình nghiên cứu trước đó. 
Và tại thời điểm ra mắt thì bài báo đã đạt được kết quả tốt nhất cho đến 
thời điểm đó.

 \item 		Tiếp theo đó nhiều bài báo lần lượt áp dụng mạng nơ ron và mạng 
hồi quy để giải quyết bài toán này như . 
 Trong đó bài báo “Variational autoencoder for collaborative” được công bố 
tại hội nghị WWW2018\cite{mvae}  bởi nhóm nghiên cứu ở Netflix và Google, 
là một bài báo nổi bật khi mà sử dụng “Variational autoencoder” để phát 
sinh các latent variables cho việc xây dựng hệ thống gợi ý sản phẩm.
 Variational Autoencoder là một mô hình Autoencoder tuy nhiên thay vì 
encoder thực hiện encode dữ liệu thành một điểm thì mô hình generative này 
sẽ phát sinh phân phối xác xuất để có thể phát sinh dữ liệu. 
 Do đó dữ liệu được phát sinh từ các phân phối xác suất sẽ đảm bảo được tính 
chất là liên tục và hoàn chỉnh có nghĩa là 2 điểm dữ liệu gần nhau sau khi 
được giải mã bởi decoder vẫn sẽ gần nhau và với một điểm dữ liệu được phát 
sinh ngẫu nhiên từ phân phối xác suất mà được tạo ra từ encoder sẽ “mang ý 
nghĩa” sau khi được giải mã.
 Từ cơ sở này thì ta có thể thấy được việc áp dụng VAEs cho bài toán gợi ý 
sản phẩm và cụ thể là áp dụng cho việc mở rộng collaborative filltering khi 
ta có thể phát hiện và khai phá các đặc trưng giữa việc tương tác với các 
item của các user từ các tương tác trong quá khứ của họ. 
 Ngoài ra thuật toán học được sử dụng trong mô hình này còn có liên kết với 
các khái niệm trong lĩnh vực lý thuyết thông tin để có thể tối ưu được kết 
quả của mô hình. 
 Với cách tiếp cận này, kết quả từ bài báo mang lại đã đánh bại 
State-of-the-art tại thời điểm đó trên các tập dữ liệu lớn từ thế giới 
thật.  
 \item	Cho đến hiện nay nhiều cải tiến trong việc áp dụng các phương pháp 
học sâu để xây dựng hệ thống gợi ý sản phẩm đã được cộng đồng nghiên cứu 
quan tâm và phát triển nhiều hơn khi nay đã có nhiều hướng tiếp cận hiện 
đại như sử dụng cơ chế attention trong bài báo \cite{bert4rec} ,, đồ thị 
tri thức \cite{kl4rec}, …  hơn và có nhiều cải tiến từ những phương pháp 
trước đó như:\cite{vamp} , … đã đạt được kết quả vượt trội đáng kể. 
\end{itemize}
    Với những thông tin mà nhóm đã tìm hiểu ở trên thì nhóm em dự định sẽ 
tập trung tìm hiểu model được đề xuất ở bài báo [5].
    Mặc dù không phải là mô hình đạt được kết quả tốt nhất hiện nay trong 
việc xây dựng hệ thống gợi ý sản tuy nhiên kiến thức nền tảng để xây 
dựng mô hình này bao phủ về mạng nơ ron, mô hình phát sinh (deep 
generative model) và kiến thức về mô hình xác suất. 
    Ngoài ra mô hình này còn thể hiện mối liên hệ với kiến thức trong lý 
thuyết thông tin (information theory) do đó việc hiểu rõ được mô hình 
này sẽ là bước đệm cho các cải tiến sau này. 
    
    
    \subsection{Kết quả dự kiến của đề tài}
        
    % Phần này nêu mô tả dự kiến các kết quả đạt được của đề tài, bao gồm 
sản phẩm, các cải tiến hoặc công trình khoa học có liên quan.
    \begin{itemize}
        \item Cài đặt lại được mô hình đề xuất trong bài báo <Citation>.
        \item Có được kết quả thí nghiệm cho thấy mô hình tự cài đặt ra được 
các kết quả như trong bài báo.
        \item Có được các kết quả thí nghiệm để thấy rõ về ưu, nhược điểm 
của mô hình.
        \item Nếu có thời gian thì có thể cài đặt và thí nghiệm thêm các cải 
tiến.
    \end{itemize}
    
    \subsection{Kế hoạch thực hiện}

\begin{tabular}{ | m{20em} | m{4cm}| m{4cm} | } 

  \hline
   \centering\textbf { Công Việc} &  \centering\textbf{Thời Gian}  & 
\begin{center}
        \textbf{Người Thực hiện}   \end{center} \\ 
  \hline
  Tìm hiểu tình hình nghiên cứu của bài toán xây dựng hệ thống gợi ý sản 
phẩm, chọn ra mô hình để tập trung tìm hiểu sâu & Tháng 01/2021 - Tháng 
02/2021 & Anh, Nhân \\ 
  \hline
  Tìm hiểu lý thuyết của mô hình đã chọn (bao gồm cả việc tìm hiểu lý thuyết 
nền tảng bên dưới &  Tháng 03/2021 & Anh, Nhân \\ 
  \hline
  Cài đặt lại từ đầu mô hình để ra được kết quả giống như trong bài báo &  
Tháng 4/2021 & Anh, Nhân \\
  \hline
  Tiến hành thí nghiệm để thấy rõ về ưu/nhược điểm của mô hình; xem xét cải 
tiến nếu có thể &  Tháng 05/2021 & Anh, Nhân \\
  \hline
 Viết cuốn và slidess &  Tháng 05/2021 - Tháng 06/2021 & Anh, Nhân \\
 \hline
\end{tabular}


   
       
    
    }
    %TÀI LIỆU TRÍCH DẪN
    %Đây là ví dụ
    \bibliographystyle{ieeetr}
    \bibliography{sample}
    \nocite{*}

    \begin{table}[h]
    \centering
        \begin{tabular}{p{7cm}p{7cm}}
        \begin{tabular}[c]{@{}c@{}}\\\textbf{XÁC NHẬN}\\\textbf{CỦA NGƯỜI 
HƯỚNG DẪN}\\ \textbf{\textit{(Ký và ghi rõ họ tên)}}\\ \\ \\ \\ \\ 
\\ThS. Trần Trung Kiên\end{tabular} 
        & \begin{tabular}[c]{@{}c@{}}\\\textbf{\textit{TP. Hồ Chí Minh, ngày 
01 tháng 03 năm 2021}}\\\textbf{NHÓM SINH VIÊN THỰC HIỆN}\\\ 
\textbf{\textit{(Ký và ghi rõ họ tên)}}\\ \\ \\ \\ \\
        \begin{tabular}{p{3cm}p{3.5cm}}
            \\ Đào Đức Anh
            & Nguyễn Thành Nhân
        \end{tabular}
        \end{tabular}
        \end{tabular}
    \end{table}

% \end{obeylines}
\end{document}


