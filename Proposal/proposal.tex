%Đây là template dùng cho đề cương đề tài tốt nghiệp
%Khoa Công nghệ Thông tin
%Trường Đại học Khoa học Tự nhiên, ĐHQG-HCM

%Liên hệ về mẫu LaTEX này: Thầy Bùi Huy Thông (bhthong@fit.hcmus.edu.vn)

\documentclass{article}[14pt]
\usepackage[utf8]{vietnam}
\usepackage{enumerate}
\usepackage{enumitem}
\usepackage{multicol}
\usepackage{listings}
\usepackage[left=2cm,right=2cm,top=2.5cm,bottom=2.5cm]{geometry}
\usepackage{verbatim}
\usepackage{graphicx}
\usepackage{url}
\usepackage{fancyhdr}
\usepackage{fancybox,framed}
\linespread{1.2}
\usepackage{lastpage}
\usepackage{floatrow}
\usepackage{floatrow}
\pagenumbering{arabic}
%\pagestyle{fancy}
\newfloatcommand{capbtabbox}{table}[][\FBwidth]

\usepackage{blindtext}
\usepackage{titlesec}
\usepackage[nottoc]{tocbibind}
\usepackage{array}
\titleformat*{\section}{\LARGE\bfseries}
\titleformat*{\subsection}{\Large\bfseries}
\titleformat*{\subsubsection}{\large\bfseries}
%\addbibresource{ref.bib}


\begin{document}
% \begin{obeylines}
    \begin{figure}[h]
        \begin{floatrow}
        \ffigbox{\includegraphics[scale = 0.1]{logo.png}}  
        {%
    
        }
        \capbtabbox{
            \begin{tabular}{l}
            \multicolumn{1}{c}{\textbf{\begin{tabular}[c]{@{}c@{}}TRƯỜNG ĐẠI 
HỌC KHOA HỌC TỰ NHIÊN\\KHOA CÔNG NGHỆ THÔNG TIN\end{tabular}}} 
\\ \\ \\
            \end{tabular}
        }
        {%
    
        }
        \end{floatrow}
    \end{figure}
    
    \begin{center}
        
        %Xác định loại đề tài tốt nghiệp tương ứng: Khóa luận, Thực tập, Đồ án
        \textbf{\Large ĐỀ CƯƠNG KHOÁ LUẬN TỐT NGHIỆP} \\ 
    \end{center}
    
    %\vspace{.5cm}
    
    \begin{center}
    %Tên đề tài phải VIẾT HOA
        
        \textbf{\huge  XÂY DỰNG HỆ THỐNG GỢI Ý SẢN PHẨM DỰA TRÊN MÔ HÌNH AUTOENCODER} \\
        
    %Tên đề tài bằng tiếng Anh (nếu có)
    \vspace{.5cm}
        \textit{\textbf{\Large (Building Recommendation System using Autoencoder model)}}
    \end{center}
    
    \vspace{.5cm}
    
    \Large
    \section{THÔNG TIN CHUNG}
    \begin{itemize}[label = {}]
        
        \item \textbf{Người hướng dẫn:} 
        %Thể hiện dạng: <Chức danh> <Họ và tên> (<Đơn vị công tác>)
        \begin{itemize}
            % \item TS. Trần Văn A (Khoa Công nghệ Thông tin)
            % \item Bà Lý Thị B (Công ty XYZ)
            \item ThS. Trần Trung Kiên (Khoa Công nghệ Thông tin)
        \end{itemize}{}
    
        
        \item \textbf{Nhóm sinh viên thực hiện:}
        
        %Thể hiện dạng: <Họ và tên sinh viên> (MSSV: )
        \begin{enumerate}
        
            \item Đào Đức Anh (MSSV: 1712270)
            \item Nguyễn Thành Nhân (MSSV: 1712631)
            
        \end{enumerate}

       %Chọn loại thích hợp
        \item \textbf{Loại đề tài:} Nghiên cứu
        
        \item \textbf{Thời gian thực hiện:} Từ \textit{1/2021} đến 
\textit{6/2021}
        
        
    \end{itemize}
    
    \section{NỘI DUNG THỰC HIỆN}
    {

    %Mỗi mục dưới đây phải viết ít nhất là 5 câu mô tả/giới thiệu.
    
    \subsection{Giới thiệu về đề tài}
    
    % Phần này giới thiệu về đề tài và ngữ cảnh thực hiện đề tài.
    
      % Phần này giới thiệu về đề tài và ngữ cảnh thực hiện đề tài.
      Bài toán xây dựng hệ thống gợi ý sản phẩm (recommendation system) được phát biểu như sau:
      \begin{itemize}
          \item Cho input là lịch sử tương tác của người dùng (user)  với 
          các sản phẩm (item) (đối với bài toán \textit{Collaborative filtering}) 
          hoặc có thêm các mô tả của người dùng và sản phẩm. 
          (đối với bài toán \textit{Content-base filtering}).
          (các sản phẩm có thể là: quảng cáo, bộ phim, văn bản để đọc, ... tùy thuộc vào ngành nghề)
          \item Yêu cầu: đưa ra tập các items (không có trong lịch sử) 
          được dự đoán là phù hợp với người dùng (tự động làm bằng máy).
      \end{itemize}
      Nếu giải quyết được bài toán này thì một ứng dụng có thể có là xây dựng 
      hệ thống hỗ trợ các dịch vụ thương mại điện tử, quảng cáo trực tuyến: 
      giúp các nhà cung cấp dịch vụ đưa ra sản phẩm/quảng cáo phù hợp với 
      thị hiếu người dùng. Một ứng dụng khác có thể có khác là hệ thống gợi ý các 
      khóa học hoặc bài giảng trong lĩnh vực giáo dục.

      Trong khóa luận này, chúng em sẽ tập trung vào bài toán 
      \textit{Collaborative filtering} vì nó giải quyết tốt 2 nhược điểm chính 
      của \textit{Content-base filtering} là: 
      (i) tận dụng được các thông tin từ các user khác và 
      (ii) trường hợp sản phẩm thiếu hoặc ít các mô tả chi tiết.
      
      Trong thời gian gần đây, các nghiên cứu chỉ ra rằng việc áp dụng các 
      mô hình phát sinh (generative model) dựa trên phương pháp học sâu 
      (deep learning), cụ thể hơn là mô hình Autoencoder trong việc cài đặt cho 
      \textit{Collaborative filtering} mang lại các kết quả tốt. 
      Và đây cũng là hướng tiếp cận chúng em chọn để tìm hiểu.

      \subsection{Mục tiêu đề tài}
      
      % Phần này mô tả mục tiêu thực hiện đề tài.
  \begin{itemize}
  \item     Hiểu rõ được tình hình nghiên cứu của bài toán xây dựng hệ thống gợi ý
   hiện nay (biết được các hướng tiếp cận phổ biến đồng thời cũng như là ý tưởng 
   và ưu nhược điểm của các hướng tiếp cận đó; ngoài ra còn nắm được các 
   thách thức và thuận lợi trong việc giải quyết bài toán gợi ý sản phẩm). Từ cơ 
   sở đó có thể chọn ra một hướng tiếp cận phù hợp để tìm hiểu sâu và thực hiện 
   cài đặt mô hình theo hướng đã chọn.
  \item      Nắm rõ các kiến thức nền tảng bên dưới (toán học, xác suất thống kê, 
  học máy, ...) của mô hình đã chọn.
  \item     Cài đặt lại mô hình  để đạt được kết quả như trong bài báo tương ứng; 
  có thể tiến hành thêm các thí nghiệm ngoài báo cáo để thấy rõ hơn về ưu nhược 
  của mô hình.
  \item     Trên cơ sở kiến thức nắm được từ mô hình có thể thực hiện các tối ưu 
  (về kết quả, tốc độ huấn luyện, ...).
  \item     Rèn luyện các kĩ năng: suy nghĩ rõ ràng, lên kế hoạch làm việc, 
  làm việc theo nhóm, khả năng trình bày báo cáo, thuyết trình, ...
  \end{itemize}

    
    \subsection{Phạm vi của đề tài}
    
    % Phần này giới hạn phạm vi thực hiện của đề tài.
    Đề tài làm với dữ liệu có các phản hồi ẩn của người dùng (implicit feedback);
     implicit feedback là phản hồi được suy ra từ hành vi của người dùng, 
    cụ thể, nếu người dùng xem bộ phim A, chúng ta có thể suy ra họ thích nó. 
    Chúng em dự kiến sẽ làm với bộ dữ liệu thường được sử 
    dụng trong lĩnh vực xây dựng  hệ thống gợi ý là MovieLens
    Về cơ bản, đề tài chỉ tìm hiểu và cài đặt lại mô hình của một bài báo uy 
tín, ngoài ra có thể thêm các thí nghiệm khác cũng như huấn luyện trên 
bộ dữ liệu khác ngoài bài báo để thấy rõ hơn về ưu, nhược điểm của mô 
hình.
    Lý do chúng em giới hạn đề tài như vậy là vì: (i) chúng em thấy việc sử dụng 
    dữ liệu implicit feedback sẽ xây dựng được một hệ thống gợi ý khách quan hơn, 
     ngoài ra, (ii) chúng em thấy riêng 
việc hiểu rõ mô hình (và các kiến thức nền tảng bên dưới) và có thể tự 
cài đặt lại đã tốn khá nhiều thời gian, và (iii) chúng em xác định là chỉ 
trên cơ sở hiểu rõ mô hình (và các kiến thức nền tảng bên dưới) thì mới 
có thể có được các cải tiến thật sự trong tương lai, cũng như là có thể 
vận dụng mô hình được cho các bài toán khác.
    Tất nhiên, trong khóa luận, nếu có đủ thời gian thì chúng em sẽ thử đề 
xuất và cài đặt các cải tiến; tuy nhiên, chúng em xác định đây không 
phải là mục đích chính.
    
    \subsection{Cách tiếp cận dự kiến}
    
    %Có thể bổ sung hình ảnh vào để làm rõ phương pháp hoặc cách tiếp cận dự 

    Trong thời đại bùng nổ thông tin hiện nay thì việc xây dựng một hệ thống gợi 
    ý sản phẩm hiệu quả có thể sẽ mang lại việc tăng lợi nhuận hay cải thiện trải 
    nghiệm người dùng cho các doanh nghiệp. Do đó việc nghiên cứu xây dựng một hệ 
    thống gợi ý sản phẩm cũng là một chủ đề được quan tâm hiện nay. Cùng với sự 
    phát triển mạnh mẽ của mạng học sâu (Deep Neuron Network) trong các lĩnh vực 
    hình ảnh, âm thanh, văn bản, … thì đã có nhiều nghiên cứu rộng rãi trong việc 
    áp dụng mạng học sâu trong bài toán gợi ý sản phẩm. Một số hướng tiếp cận áp 
    dụng mạng học sâu ta có thể điểm qua như: Mạng nơ ron nhiều tầng (Multi Layer 
    Perceptron), mạng nơ ron hồi quy (Recurrent Neuron Network), Autoencoder, cơ chế attention, … 
    Trong khóa luận này, chúng em sẽ nghiên cứu và tìm hiểu hướng tiếp cận sử 
    dụng Autoencoder trong việc xây dựng hệ thống gợi ý sản phẩm. Mô hình 
    Autoencoder là một mạng nơ ron bao gồm 2 phần: (1) Encoder là một mạng nơ ron 
    có nhiệm vụ biểu diễn dữ liệu ở chiều không gian ban đầu thành một điểm dữ 
    liệu ở chiều không gian thấp hơn. Điểm dữ liệu ở chiều không gian mới này có 
    thể dùng để biểu diễn các đặc trưng ẩn (latent variable) của dữ liệu ban đầu; 
    (2) Decoder là một mạng nơ ron khác dùng để tái cấu trúc lại dữ liệu ban đầu 
    từ các đặc trưng ẩn.
    Dưới đây sẽ trình bày một số hướng tiếp cận sử dụng mô hình Autoencoder để 
    giải quyết bài toán gợi ý sản phẩm mà chúng em đã tìm hiểu được cho đến thời 
    điểm hiện tại, cũng như là mô hình mà chúng em dự kiến sẽ chọn để tập trung 
    tìm hiểu sâu:
    
    
    
\begin{itemize}
    \item   “AutoRec: Autoencoders Meet Collaborative Filtering”\cite{autorec} được đề xuất bởi 
    nhóm tác giả từ trường đại học quốc gia Australia  ở hội nghị WWW2015, là một 
    trong những bài báo đầu tiên áp dụng mô hình Autoencoder để xây dựng một hệ 
    thống gợi ý sản phẩm. Mô hình AutoRec được đề xuất bởi nhóm tác giả nhậninput 
    là vector tương tác giữa user và item, mô hình sẽ chiếu input lên một không 
    gian có số chiều thấp hơn để biểu diễn các đặc trưng ẩn của dữ liệu, và sau 
    đó chúng sẽ được tái cấu trúc lại với mục đích để dự đoán những tương tác bị 
    trống trước đó. Mô hình học bằng cách tối thiểu độ lỗi trong việc tái cấu 
    trúc dữ liệu ban đầu từ các đặc trưng ẩn và để tránh overfitting thì mô hình 
    có áp dụng kĩ thuật chính quy hóa (regularization) trong việc huấn luyện mạng 
    nơ ron trong Autoencoder. Trong bài báo này, tác giả đã thực nghiệm và cho 
    kết quả đáng mong đợi khi mô hình đã đánh bại được mô hình nổi bật trong các 
    hệ thống gợi ý sản phẩm theo hướng Collaborative Filtering là Matrix 
    Factorization. Bên cạnh đó, tác giả còn có các thử nghiệm mang lại những dấu 
    hiệu tích cực trong việc áp dụng mạng nơ ron khi tăng số tầng ẩn hay kích 
    thước của các tầng ẩn trong mạng thì kết quả được cải thiện theo. Điều này đã 
    dẫn đến việc áp dụng các phương pháp học sâu trong việc xây dựng hệ thống gợi 
    ý sản phẩm được quan tâm và nghiên cứu nhiều hơn trong cộng đồng nghiên cứu 
    khoa học.
    \item Sau đó không lâu đã có nhiều bài báo được xuất bản dựa trên mô hình 
    Autoencoder để xây dựng hệ thống gợi ý sản phẩm. Một trong những bài báo cáo 
    khoa học nổi bật đó là “Hybrid Recommender System based on Autoencoders” 
    \cite{cfn} của Florian Strub cùng các cộng sự, nhóm tác giả đã đề xuất mô 
    hình có tên là CFN, đây được xem như là một mô hình mở rộng của Autorec. 
    Input của mô hình tương đồng với Autorec, tuy nhiên nhóm tác giả đề xuất 2 
    điểm cải thiện cho mô hình đó là: (1) áp dụng kĩ thuật khử nhiễu 
    (denoising); (2) ngoài dữ liệu tương tác của user và item thì tác giả còn 
    kết hợp với những thông tin khác như dữ liệu về user hay dữ liệu của item để 
    giải quyết hai vấn đề thường gặp trong hệ thống gợi ý sản phẩm là dữ liệu 
    thưa\footnote{Dữ liệu ban đầu có nhiều trường bị trống, hay tương tác giữa 
    user và item quá ít} và cold start\footnote{khi một user mới hoặc item mới 
    được thêm vào hệ thống thì sẽ có rất ít tương tác của user hay đối với item 
    đó thì sẽ khó để tìm được sự tương đồng giữa các user trong việc tương tác 
    với item đó}; Bên cạnh đó có thể kể đến “Collaborative Denoising 
    Auto-Encoders for Top-N Recommender Systems” \cite{cde} nói về mô hình CDE 
    được đề xuất bởi nhóm nghiên cứu tại trường đại học Beihang, Trung Quốc, 
    cũng là một bài báo nổi bật trong việc áp dụng mô hình Autoencoder, mô hình 
    này sử dụng một biến thể của Autoencoder là Denoising Autoencoder để xây 
    dựng hệ thống gợi ý sản phẩm, điểm khác biệt của mô hình CDE là input của mô 
    hình sẽ được thêm nhiễu nhằm giúp mô hình tránh tình trạng overfitting. Và 
    theo bài báo thì CDE có được kết quả vượt trội hơn so với những state of the 
    art tại thời điểm được đề xuất.
\item	Đặc biệt mô hình Multi-VAE được đề xuất bởi nhóm nghiên cứu của Netflix và 
Google tại hội nghị WWW2018 với tiêu đề "Variational Autoencoders for 
Collaborative Filtering" \cite{mvae}, là một trong những bài báo đáng chú ý trong việc xây 
dựng hệ thống gợi ý sản phẩm, mô hình được cải thiện so với những mô hình trước 
đó bằng cách áp dụng Variational Autoencoders (VAEs), là một biến thể của 
Autoencoder. Ở Variational Autoencoder thì latent variables sẽ là phân phối xác 
suất thay vì một điểm dữ liệu trong chiều không gian thấp hơn như Autoencoder 
hay Denoising Autoencoder. Do đó dữ liệu sau khi tái cấu trúc sẽ đảm bảo được 
tính chất “liên tục” và “toàn vẹn” cho dữ liệu. Điều đó có nghĩa là 2 điểm dữ 
liệu gần nhau trong không gian dữ liệu ban đầu nếu gần nhau thì sau khi tái cấu 
trúc sẽ gần nhau (tính liên tục) và việc phát sinh ngẫu nhiên dữ liệu từ phân 
phối xác suất ẩn sẽ đảm bảo được tính toàn vẹn của dữ liệu (tính toàn vẹn). Từ 
cơ sở này ta có thể thấy được áp dụng VAEs cho bài toán gợi ý sản phẩm và cụ 
thể là theo hướng Collaborative Filtering là một lựa chọn phù hợp khi sự tương 
đồng giữa các user trong việc tương tác với các item sẽ được khám phá tốt hơn. 
Do đó ta có thể xây dựng một hệ thống gợi ý sản phẩm hiệu quả. Ngoài ra trong 
bài báo này, tác giả còn đề xuất việc sử dụng Multinomial Loglikelihood là phù 
hợp cho việc mô hình hóa dữ liệu implicit feedback thay vì những hàm phân phối 
phổ biến như logistic likelihood hay Gaussian Likelihood. Và mô hình đã mang 
lại một kết quả vượt bậc so với những mô hình trước đó.

\end{itemize}
Với những thông tin mà nhóm đã tìm hiểu ở trên thì nhóm em dự định sẽ tập trung 
tìm hiểu model được đề xuất ở bài báo . Mặc dù không phải là mô hình đạt 
được kết quả tốt nhất hiện nay trong việc xây dựng hệ thống gợi ý sản phẩm, tuy 
nhiên kiến thức nền tảng để xây dựng mô hình này bao phủ về mạng nơ ron, mô 
hình phát sinh (deep generative model) và kiến thức về mô hình xác suất. Ngoài 
ra mô hình này còn thể hiện mối liên hệ với kiến thức trong lý thuyết thông tin 
(information theory) do đó việc hiểu rõ được mô hình này sẽ là bước đệm cho các 
cải tiến sau này.
    
    
    \subsection{Kết quả dự kiến của đề tài}
        
    % Phần này nêu mô tả dự kiến các kết quả đạt được của đề tài, bao gồm sản phẩm, các cải tiến hoặc công trình khoa học có liên quan.
    \begin{itemize}
        \item Cài đặt lại được mô hình đề xuất trong bài báo \cite{mvae}.
        \item Có được kết quả thí nghiệm cho thấy mô hình tự cài đặt ra được 
các kết quả như trong bài báo.
        \item Có được các kết quả thí nghiệm để thấy rõ về ưu, nhược điểm 
của mô hình.
        \item Nếu có thời gian thì có thể cài đặt và thí nghiệm thêm các cải 
tiến.
    \end{itemize}
    
    \subsection{Kế hoạch thực hiện}

\begin{tabular}{ | m{20em} | m{4cm}| m{4cm} | } 

  \hline
   \centering\textbf { Công Việc} &  \centering\textbf{Thời Gian}  & 
\begin{center}
        \textbf{Người Thực hiện}   \end{center} \\ 
  \hline
  Tìm hiểu tình hình nghiên cứu của bài toán xây dựng hệ thống gợi ý sản 
phẩm, chọn ra mô hình để tập trung tìm hiểu sâu & Tháng 01/2021 - Tháng 
02/2021 & Anh, Nhân \\ 
  \hline
  Tìm hiểu lý thuyết của mô hình đã chọn (bao gồm cả việc tìm hiểu lý thuyết 
nền tảng bên dưới &  Tháng 03/2021 & Anh, Nhân \\ 
  \hline
  Cài đặt lại từ đầu mô hình để ra được kết quả giống như trong bài báo &  
Tháng 4/2021 & Anh, Nhân \\
  \hline
  Tiến hành thí nghiệm để thấy rõ về ưu/nhược điểm của mô hình; xem xét cải 
tiến nếu có thể &  Tháng 05/2021 & Anh, Nhân \\
  \hline
 Viết cuốn và slidess &  Tháng 05/2021 - Tháng 06/2021 & Anh, Nhân \\
 \hline
\end{tabular}


   
       
    
    }
    %TÀI LIỆU TRÍCH DẪN
    %Đây là ví dụ
    \bibliographystyle{ieeetr}
    \bibliography{sample}
    \nocite{*}

    \begin{table}[h]
    \centering
        \begin{tabular}{p{7cm}p{7cm}}
        \begin{tabular}[c]{@{}c@{}}\\\textbf{XÁC NHẬN}\\\textbf{CỦA NGƯỜI 
HƯỚNG DẪN}\\ \textbf{\textit{(Ký và ghi rõ họ tên)}}\\ \\ \\ \\ \\ 
\\ThS. Trần Trung Kiên\end{tabular} 
        & \begin{tabular}[c]{@{}c@{}}\\\textbf{\textit{TP. Hồ Chí Minh, ngày 
01 tháng 03 năm 2021}}\\\textbf{NHÓM SINH VIÊN THỰC HIỆN}\\\ 
\textbf{\textit{(Ký và ghi rõ họ tên)}}\\ \\ \\ \\ \\
        \begin{tabular}{p{3cm}p{3.5cm}}
            \\ Đào Đức Anh
            & Nguyễn Thành Nhân
        \end{tabular}
        \end{tabular}
        \end{tabular}
    \end{table}

% \end{obeylines}
\end{document}


