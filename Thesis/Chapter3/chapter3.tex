\chapter{Mô hình ``Variational Auto-Encoder'' cho bài toán xây dựng hệ thống gợi ý''}
\label{Chapter3}
\textit{Chương này trình bày về những đóng góp của luận văn. 
Ở đây, Chúng tôi phân tích hai loại dữ liệu phản hồi chính từ người dùng là: 
``explicit feedback'' và ``implicit feedback''.
Đặc biệt, chúng tôi tập trung nghiên cứu mở rộng mô hình ``Variational Auto-Encoders'' 
cho implicit feedback với hàm lỗi là ``Multinomial Log-likelihood'' ở hàm mục tiêu.
Chúng tôi gọi ``Variational Auto-Encoders'' với hàm lỗi như vậy là ``Mul-VAE''. 
Đóng góp của chúng tôi là làm rõ Mul-VAE ở hai điểm:     
\begin{itemize}
	\item Tính xếp hạng: Chúng tôi chỉ ra điểm phù hợp của Multinomial Log-likelihood cho bài toán xây dựng hệ thống gợi ý sản phẩm so với các hàm Log-likelihood thông dụng khác.
	\item KL-Annealing: chúng tôi cũng đưa ra một cách ``heuristic'' nhằm lựa chọn siêu tham số của mô hình Mul-VAEs.
    \end{itemize}
}
\section{Dữ liệu phản hồi của người dùng trong bài toán xây dựng hệ thống gợi ý sản phẩm}
    \subsection{Dữ liệu phản hồi cụ thể ``explicit feedback''}
    \subsection{Dữ liệu phản hồi ngầm ``implicit feedback''}

\section{``Multinomial log-likelihood'' cho bài toán xây dựng hệ thống gợi ý}
\section{``Mul-VAEs''}
    \subsection{Quá trình huấn luyện mô hình}
    \subsection{Quá trình phát sinh gợi ý}
\section{Vấn đề KL-Vanishing trong việc huấn luyện ``Variational Auto-Encoder''}
    \subsection{Phương pháp ``KL-Annealing''}