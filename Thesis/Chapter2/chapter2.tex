\chapter{Kiến thức nền tảng}
\label{Chapter2}
\textit{Tong chương này, đầu tiên chúng tôi sẽ trình bày về mô hình 
``Auto-Encoders'', một mạng nơ-ron được dùng để học đặc trưng ẩn dựa trên
phương pháp học không giám sát.
Sau đó, chúng tôi giới thiệu và trình bày về nền tảng xác suất 
của ``Variational Auto-encoders'' (VAEs) và lợi ích mang lại của 
mô hình này so với ``Auto-Encoder'' trong tác vụ học đặc trưng ẩn; 
Những điểm lợi này chính là lý do mà chúng tôi tập trung nghiên cứu VAEs. 
Bên cạnh đó, chúng tôi sẽ trình bày về ``Maximum Likelihood Estimation'', 
một phương pháp dùng để đánh giá các tham số của mô hình, đại diện cho các tham số 
của các phân phối xác suất dựa trên dữ liệu huấn luyện. 
Chương này đặc biệt là phần về ``Variational Auto-Encoders'' 
cung cấp những kiến thức nền tảng để có thể hiểu rõ về những đề xuất 
của chúng tôi ở chương kế tiếp.}


% \textit{Bên cạnh đó, chúng tôi sẽ trình bày về ``Log-likelihood fuctions'' - 
% là phép đo mức độ học của mô hình dựa trên các dữ liệu ta quan sát được 
% cũng như các lựa chọn Log-likelihood fucntion cho các bài toán học máy hiện nay.}
% 7->10; 15->20; 15->20; 10->15; 2 (~60)
\section{Mô hình rút trích đặc trưng ``Auto-Encoder''}
\label{chap2/sec1}
    \subsection{``Undercomplete Auto-Encoder''}
    \label{chap2/subsec11}
    \subsection{``Denoising Auto-Encoder''}
    \label{chap2/subsec12}

\section{``Variational Auto-Encoder''}
    \subsection{Nền tảng xác suất}
        \subsubsection{ ``Maximum Likelihood Estimation''}
    \subsection{Phương pháp ``Variational Inference'' }
    \subsection{Độ sai biệt ``Kullback-Leiber Devergence'' giữa hai phân phối xác suất}
