
\chapter*{TÓM TẮT}
\label{tomtat}

Hệ thống gợi ý sản phẩm được sinh ra nhằm giúp giúp người dùng giải quyết các khó khăn khi họ muốn tìm kiếm một nội dung phù hợp trên Internet, khi mà dữ liệu trên Internet ngày nay là khổng lồ. Hệ thống gợi ý sản phẩm hiện nay đã và đang đóng góp vai trò quan trọng trong các doanh nghiệp, với mục tiêu nâng cao trải nghiệm người dùng cũng như thu hút khách hàng. Có thể kể đến một số doanh nghiệp lớn như Netflix, Amazon, Google được ảnh hưởng một cách tích cực bởi hệ thống gợi ý.

Bởi những ứng dụng của nó trong thực tiễn, trong thời gian gần đây bài toán xây dựng hệ thống gợi ý sản phẩm đang là một vấn đề được cộng đồng nghiên cứu khoa học quan tâm. 

Với việc bùng nổ dữ liệu trên Internet hiện nay, việc xây dựng một hệ thống gợi ý tận dụng được dữ liệu tương tác từ người dùng khác trở nên hiệu quả hơn so với cách tiếp cận truyền thống là đưa ra gợi ý có tính tương đồng với cản sản phẩm trước đó. Đây cũng chính là lí do khóa luận tập trung tìm hiểu hướng tiếp cận ``Collaborative Filtering'' (hướng tiếp cận đưa ra gợi ý dựa trên dữ liệu tương tác từ những người dùng khác). Một bài báo nổi bật trong việc xây dựng hệ thống gợi ý sản phẩm với hướng tiếp cận này là ``Variational Autoencoder for Collaborative Filtering'' \cite{mvae} được giới thiệu bởi tác giả Liang cùng các cộng sự tại hội nghị ``International World Wide Web Conference Committee 2018''; mô hình được bài báo đề xuất bao gồm hai phần:
\begin{itemize}
    \item Tiến trình suy diễn: kết hợp giữa mạng nơ-ron encoder trong mô hình ``Auto-Encoder'' và phương pháp ``Variational Inference'' trong lĩnh vực xác suất thống kê nhằm suy diễn dữ liệu tương tác của người dùng thành véc-tơ đặc trưng ẩn đại diện cho sở thích của người dùng
    \item Tiến trình phát sinh: mạng nơ-ron decoder dựa trên mô hình ``Auto-Encoder'' nhằm tái tạo lại tương tác của người dùng từ đặc trưng ẩn.
\end{itemize}

Ưu điểm của mô hình này này là dựa phương pháp ``Variational Inference'', một phương pháp dùng để suy diễn dữ liệu hiệu quả. Mô hình sẽ ước lượng một phân phối xác suất cho đặc trưng ẩn thay vì một véc-tơ cố định; nghĩa là không có sự chắc chắn trong đó, sự không chắc chắn này phần nào giúp hạn chế việc ``overfitting''. Trong thực tế, dữ liệu tương tác của người dùng thường là dữ liệu thưa (đa số các phần tử trong véc-tơ này là 0) vì mỗi người dùng chỉ tương tác với một lượng nhỏ sản phẩm trên toàn bộ hệ thống, do đó việc suy diễn trở nên hiệu quả hơn. 

Với mục tiêu đưa ra ``top-N'' sản phẩm phù hợp nhất với người dùng, các tác giả giới thiệu thêm hàm lỗi ``Multinomial log-likelihood'', với tính chất trả về một giá trị xác suất cho mỗi sản phẩm, và tổng giá trị xác suất trên toàn bộ sản phẩm là 1. Các sản phẩm sẽ phải ``cạnh tranh'' với nhau để có được xác suất cao hơn. 

Kết quả đạt được của khóa luận là cài đặt lại được mô hình tìm hiểu với kết quả tương đương bài báo gốc \cite{mvae}. Khóa luận cũng tiến hành thêm một số thí nghiệm nhằm đánh giá rõ hơn về tính chất của mô hình.

% Hệ thống gợi ý sản phẩm đã và đang đóng góp vai trò quan trọng trong các doanh nghiệp. Có thể nói hệ thống gợi ý sản phẩm đóng vai trò quan trọng trong sự phát triển của doanh nghiệp. Cũng như góp phần không nhỏ nhằm tăng trải nghiệm và tiết kiệm thời gian cho người dùng. 
% Hiện nay, hệ thống gợi ý đang là một vấn đề được cộng đồng nghiên cứu khoa học quan tâm trong thời gian gần đây bởi những ứng dụng của nó trong thực tiễn. 