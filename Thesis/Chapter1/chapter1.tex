\chapter{Giới thiệu}
\label{Chapter1}



% Ngôn ngữ để viết và trình bày báo cáo khóa luận tốt nghiệp, đồ án tốt nghiệp, thực tập tốt nghiệp (sau đây gọi chung là báo cáo) là tiếng Việt hoặc tiếng Anh. 
% Trường hợp chọn ngôn ngữ tiếng Anh để viết và trình bày báo cáo,  sinh viên cần có đơn đề nghị, được cán bộ hướng dẫn (CBHD) đồng ý và nộp cho bộ phận Giáo vụ của Khoa vào thời điểm đăng ký đề tài để xin ý kiến.
% Báo cáo viết và trình bày bằng tiếng Anh phải có bản tóm tắt viết bằng tiếng Việt.


%Tóm tắt luận văn được trình bày nhiều nhất trong 24 trang in trên hai mặt giấy, cỡ chữ Times New Roman 11 của hệ soạn thảo Winword hoặc phần mềm soạn thảo Latex đối với các chuyên ngành thuộc ngành Toán.

%Mật độ chữ bình thường, không được nén hoặc kéo dãn khoảng cách giữa các chữ.
%Chế độ dãn dòng là Exactly 17pt.
%Lề trên, lề dưới, lề trái, lề phải đều là 1.5 cm.
%Các bảng biểu trình bày theo chiều ngang khổ giấy thì đầu bảng là lề trái của trang.
%Tóm tắt luận án phải phản ảnh trung thực kết cấu, bố cục và nội dung của luận án, phải ghi đầy đủ toàn văn kết luận của luận án.
%Mẫu trình bày trang bìa của tóm tắt luận văn (phụ lục 1).
Với việc bùng nổ dữ liệu trên mạng Internet hiện nay, con người khó có thể nắm bắt được hết tất cả các thông tin.
Do đó việc tìm kiếm dữ liệu, thông tin phù hợp với nhu cầu của bản thân dần trở nên khó khăn.
Việc có một hệ thống gợi ý hỗ trợ chúng ta trong việc tìm kiếm thông tin là cực kỳ hữu ích. 
Một hệ thống gợi ý sản phẩm là một bài toán trong lĩnh vực khai thác dữ liệu và học máy. 
Hệ thống gợi ý được xây dựng để dự đoán những sản phẩm phù hợp với người dùng, đặc biệt hiện nay việc đưa ra quyết định khi mà có quá nhiều lựa chọn dành cho người dùng là không hề dễ dàng. 
Điều này dẫn đến vai trò của một hệ thống gợi ý ngày càng quan trọng hơn, không chỉ hỗ trợ người dùng đưa ra quyết định mà còn đóng góp trong việc phát triển doanh nghiệp khi mà việc thu hút khách hàng và nâng cao trải nghiệm người dùng sẽ phù thuộc vào một hệ thống gợi ý sản phẩm hiệu quả. 
Có rất nhiều lĩnh vực cần xây dựng hệ thống gợi ý có thể kể đến như thương mại điện thử, các nền tàng cung cấp các dịch vụ đa phương tiện (âm thanh, hình ảnh, video, ...), mạng xã hội (Facebook, tweeter, linkedin, ... ).  
Theo số liệu tổng hợp được thì 75\% phim được thuê trên Netflix - một nền tảng cung cấp video nổi tiếng hiện nay đến từ hệ thống gợi ý; 38\% lượt click từ người dùng GOOGLE cũng đến từ hệ thống gợi ý và Amazon - một nền tảng mua bán trực tuyến mà 35\% sản phẩm được bán thông qua hệ thống gợi ý sản phẩm. 

Để xây dựng một hệ thống gợi ý sản phẩm ta có các hướng tiếp cận là 
