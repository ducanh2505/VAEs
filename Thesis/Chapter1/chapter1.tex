\chapter{Ngôn ngữ}
\label{Chapter1}

Ngôn ngữ để viết và trình bày báo cáo khóa luận tốt nghiệp, đồ án tốt nghiệp, thực tập tốt nghiệp (sau đây gọi chung là báo cáo) là tiếng Việt hoặc tiếng Anh. 
Trường hợp chọn ngôn ngữ tiếng Anh để viết và trình bày báo cáo,  sinh viên cần có đơn đề nghị, được cán bộ hướng dẫn (CBHD) đồng ý và nộp cho bộ phận Giáo vụ của Khoa vào thời điểm đăng ký đề tài để xin ý kiến.
Báo cáo viết và trình bày bằng tiếng Anh phải có bản tóm tắt viết bằng tiếng Việt.


%Tóm tắt luận văn được trình bày nhiều nhất trong 24 trang in trên hai mặt giấy, cỡ chữ Times New Roman 11 của hệ soạn thảo Winword hoặc phần mềm soạn thảo Latex đối với các chuyên ngành thuộc ngành Toán.

%Mật độ chữ bình thường, không được nén hoặc kéo dãn khoảng cách giữa các chữ.
%Chế độ dãn dòng là Exactly 17pt.
%Lề trên, lề dưới, lề trái, lề phải đều là 1.5 cm.
%Các bảng biểu trình bày theo chiều ngang khổ giấy thì đầu bảng là lề trái của trang.
%Tóm tắt luận án phải phản ảnh trung thực kết cấu, bố cục và nội dung của luận án, phải ghi đầy đủ toàn văn kết luận của luận án.
%Mẫu trình bày trang bìa của tóm tắt luận văn (phụ lục 1).